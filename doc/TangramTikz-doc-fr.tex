% !TeX TXS-program:compile = txs:///arara
% arara: pdflatex: {shell: yes, synctex: no, interaction: batchmode}
% arara: pdflatex: {shell: yes, synctex: no, interaction: batchmode} if found('log', '(undefined references|Please rerun|Rerun to get)')

\documentclass{article}
\usepackage[french]{babel}
\usepackage[utf8]{inputenc}
\usepackage[T1]{fontenc}
\usepackage{TangramTikz}
%\usepackage[upright]{fourier}
%\usepackage[scaled=0.875]{helvet}
%\renewcommand\ttdefault{lmtt}
%\usepackage{cabin}
\usepackage{amsmath,amssymb}
\usepackage{fontawesome5}
\usepackage{enumitem}
\usepackage{tabularray}
\usepackage{multicol}
\usepackage{fancyvrb}
\usepackage{fancyhdr}
\fancyhf{}
\renewcommand{\headrulewidth}{0pt}
\lfoot{\sffamily\small [TangramTikz]}
\cfoot{\sffamily\small - \thepage{} -}
\rfoot{\hyperlink{matoc}{\small\faArrowAltCircleUp[regular]}}

%\usepackage{hvlogos}
\usepackage{hologo}
\providecommand\tikzlogo{Ti\textit{k}Z}
\providecommand\TeXLive{\TeX{}Live\xspace}
\providecommand\PSTricks{\textsf{PSTricks}\xspace}
\let\pstricks\PSTricks
\let\TikZ\tikzlogo
\newcommand\TableauDocumentation{%
	\begin{tblr}{width=\linewidth,colspec={X[c]X[c]X[c]X[c]X[c]X[c]},cells={font=\sffamily}}
		{\huge \LaTeX} & & & & &\\
		& {\huge \hologo{pdfLaTeX}} & & & & \\
		& & {\huge \hologo{LuaLaTeX}} & & & \\
		& & & {\huge \TikZ} & & \\
		& & & & {\huge \TeXLive} & \\
		& & & & & {\huge \hologo{MiKTeX}} \\
	\end{tblr}
}

\usepackage{hyperref}
\urlstyle{same}
\hypersetup{pdfborder=0 0 0}
\usepackage[margin=1.5cm]{geometry}
\setlength{\parindent}{0pt}
\definecolor{LightGray}{gray}{0.9}

\def\TPversion{0.1.5}
\def\TPdate{25 Février 2023}

\usepackage[most]{tcolorbox}
\tcbuselibrary{minted}
\NewTCBListing{PresentationCode}{ O{blue} m }{%
	sharp corners=downhill,enhanced,arc=12pt,skin=bicolor,%
	colback=#1!1!white,colframe=#1!75!black,colbacklower=white,%
	attach boxed title to top right={yshift=-\tcboxedtitleheight},title=Code \LaTeX,%
	boxed title style={%
		colframe=#1!75!black,colback=#1!15!white,%
		,sharp corners=downhill,arc=12pt,%
	},%
	fonttitle=\color{#1!90!black}\itshape\ttfamily\footnotesize,%
	listing engine=minted,minted style=colorful,
	minted language=tex,minted options={tabsize=4,fontsize=\footnotesize,autogobble},
	#2
}

\newcommand\Cle[1]{{\bfseries\sffamily\textlangle #1\textrangle}}

\begin{document}

\pagestyle{fancy}

\thispagestyle{empty}

\vspace{2cm}

\begin{center}
	\begin{minipage}{0.75\linewidth}
	\begin{tcolorbox}[colframe=yellow,colback=yellow!15]
		\begin{center}
			\begin{tabular}{c}
				{\Huge \texttt{TangramTikz [fr]}}\\
				\\
				{\LARGE Des tangrams, en Ti\textit{k}Z}, \\
				\\
				{\LARGE avec solution et/ou couleur.} \\
			\end{tabular}
			
			\medskip
			
			{\small \texttt{Version \TPversion{} -- \TPdate}}
		\end{center}
	\end{tcolorbox}
\end{minipage}
\end{center}

\vspace{0.5cm}

\begin{center}
	\begin{tabular}{c}
	\texttt{Cédric Pierquet}\\
	{\ttfamily c pierquet -- at -- outlook . fr}\\
	\texttt{\url{https://github.com/cpierquet/TangramTikz}}
\end{tabular}
\end{center}

\vspace{0.5cm}

{$\blacktriangleright$~~Des commandes pour afficher des Tangrams prédéfinis.}

\smallskip

{$\blacktriangleright$~~Possibilité de créer un Tangram, avec le placement des pièces.}

\smallskip

{$\blacktriangleright$~~Idée(s) venant de \url{https://tex.stackexchange.com/questions/407449/typesetting-tangram-figures-in-latex}}

\vspace{1cm}

\begin{center}
	\tikz {\pic[TangPuzz={blue}] at (0,0) {TangBigTri} ;}~~
	\tikz {\pic[TangPuzz={orange}] at (0,0) {TangBigTri} ;}~~
	\tikz {\pic[TangPuzz={purple}] at (0,0) {TangMedTri} ;}~~
	\tikz {\pic[TangPuzz={yellow}] at (0,0) {TangSqua} ;}~~
	\tikz {\pic[TangPuzz={green}] at (0,0) {TangSmalTri} ;}~~
	\tikz {\pic[TangPuzz={cyan}] at (0,0) {TangSmalTri} ;}~~
	\tikz {\pic[TangPuzz={magenta}] at (0,0) {TangPara} ;}~~
	
	\vspace*{1cm}
	
	\TangramTikz[Couleur=orange]<scale=1>{Maison}
	\TangramTikz[Correction]<scale=1>{Maison}
	\TangramTikz[CorrectionCouleur]<scale=1>{Maison}
	\TangramTikz[ListeCouleurs={blue,red,black,orange,purple},CorrectionCouleur]<scale=1>{Maison}
\end{center}

\vspace{0.5cm}

%\hfill{}\textit{Merci aux membres du groupe \faFacebook{} du \og Coin \LaTeX{} \fg{} pour leur aide et leurs idées !}

%\hfill{}\textit{Merci à Denis Bitouzé et à Patrick Bideault pour leurs retours et idées !}

\vfill

\hrule

\medskip

\TableauDocumentation

\medskip

\hrule

\medskip

\newpage

\phantomsection
\hypertarget{matoc}{}

\tableofcontents

\newpage

\part{Introduction}

\section{Le package TangramTikz}

\subsection{Source}

Certaines idées viennent de \url{https://tex.stackexchange.com/questions/407449/typesetting-tangram-figures-in-latex}, avec une proposition de Andrew Stacey.

\smallskip

Le package a ensuite été \textit{construit} et \textit{modestement enrichi} autour de styles et méthodes proposées par Andrew Stacey.

\subsection{Chargement du package, packages utilisés}

Le package \textsf{TangramTikz} se charge dans le préambule via la commande :

\begin{PresentationCode}{listing only}
\usepackage{TangramTikz}
\end{PresentationCode}

Il est compatible avec les compilations usuelles en \textsf{latex}, \textsf{pdflatex}, \textsf{lualatex} ou \textsf{xelatex}.

\medskip

Il charge les packages et librairies suivantes :

\begin{itemize}
	\item \texttt{tikz} avec les librairies \Cle{calc} et \Cle{shapes.geometric} ;
	\item \texttt{xstring}, \texttt{xparse}, \texttt{simplekv} et \texttt{listofitems}.
\end{itemize}

\subsection{\og Philosophie \fg{} du package}

L'idée est de proposer, grâce à \TikZ, des \textsf{commandes} pour présenter un jeu de Tangram :

\begin{itemize}
	\item sous forme du puzzle avec pièces \textit{pleines} ;
	\item sous forme du puzzle avec pièces \textit{avec une petite bordure} ;
	\item sous forme du puzzle coloré avec pièces \textit{avec une petite bordure}.
\end{itemize}

\begin{PresentationCode}{listing only}
%commande autonome pour afficher un Tangram
\TangramTikz[clés]<options tikz>{nom_du_tangram}
\end{PresentationCode}

Il est également proposé un \textsf{environnement} ainsi qu'une \textsf{commande} pour construire soi-même le puzzle, en plaçant \textit{manuellement} les pièces.

\begin{PresentationCode}{listing only}
%environnement, avec clés en français, et placement des pièces
\begin{EnvTangramTikz}[clés]<options tikz>
	%placement des pièces
	\PieceTangram[clés]<options pic>(decalH,decalV){TangGrandTri}
	\PieceTangram[clés]<options pic>(decalH,decalV){TangGrandTri}
	\PieceTangram[clés]<options pic>(decalH,decalV){TangMoyTri}
	\PieceTangram[clés]<options pic>(decalH,decalV){TangPetTri}
	\PieceTangram[clés]<options pic>(decalH,decalV){TangPetTri}
	\PieceTangram[clés]<options pic>(decalH,decalV){TangCar}
	\PieceTangram[clés]<options pic>(decalH,decalV){TangPara}
	%\filldraw[black] (0,0) circle[radius=4pt] ; %repère pour les pièces
\end{EnvTangramTikz}
\end{PresentationCode}

\pagebreak

\part{Le fonctionnement}

\section{Fonctionnement \og manuel \fg}

\subsection{Les pièces du Tangram}

Un Tangram est composé de 7 pièces (visibles sur la page de garde):

\begin{itemize}
	\item 2 grands triangles isocèles rectangles ;
	\item 1 triangle isocèle rectangle ;
	\item 2 petits triangles isocèles rectangles ;
	\item 1 carré ;
	\item 1 parallélogramme.
\end{itemize}

Chacune des pièces qui compose le Tangram est définie en langage \tikzlogo, sous forme d'un \texttt{pic} autonome.

\medskip

Le schéma suivant propose de visualiser les (5) pièces différentes :

\begin{itemize}
	\item avec leur \textcolor{purple}{\texttt{nom}} en code \tikzlogo{} ;
	\item avec leur \textit{orientation} initiale ;
	\item leur \textcolor{red}{\textit{origine}} initiale ;
	\item leurs \textcolor{blue}{\textit{dimensions}} utiles (qui sont données en \textit{unité}).
\end{itemize}

\begin{center}
	\begin{tikzpicture}[scale=1.25]
		\draw[thin,lightgray!50] (-1,-1) grid (3,3) ;
		\PieceTangram{TangGrandTri} \filldraw[red] (0,0) circle[radius=2pt] ;
		\draw[thick,<->,>=latex] (0,-0.25)--(2,-0.25) node[blue,scale=1.5,midway,below,font=\large\sffamily] {2} ;
		\draw[thick,<->,>=latex] (2.25,0)--++(0,2) node[blue,scale=1.5,midway,right,font=\large\sffamily] {2} ;
		\draw[thick,<->,>=latex] (-0.15,0.15)--++(45:{sqrt(8)}) node[blue,scale=1.5,midway,sloped,above,font=\large\sffamily] {$\mathsf{2\sqrt{2}}$} ;
		\draw (1,3) node[scale=1.5,purple,below=1pt,font=\Large\ttfamily] {TangGrandTri} ;
	\end{tikzpicture}
	~
	\begin{tikzpicture}[scale=1.25]
		\draw[thin,lightgray!50] (-1,-1) grid (3,3) ;
		\PieceTangram{TangMoyTri} \filldraw[red] (0,0) circle[radius=2pt] ;
		\draw[thick,<->,>=latex] (0,-0.25)--(2,-0.25) node[blue,scale=1.5,midway,below,font=\large\sffamily] {2} ;
		\draw[thick,<->,>=latex] (-0.15,0.15)--++(45:{sqrt(2)}) node[blue,scale=1.5,midway,sloped,above,font=\large\sffamily] {$\mathsf{\sqrt{2}}$} ;
		\draw[thick,<->,>=latex] (2.15,0.15)--++(135:{sqrt(2)}) node[blue,scale=1.5,midway,sloped,above,font=\large\sffamily] {$\mathsf{\sqrt{2}}$} ;
		\draw (1,3) node[purple,scale=1.5,below=1pt,font=\Large\ttfamily] {TangMoyTri} ;
	\end{tikzpicture}
	~
	\begin{tikzpicture}[scale=1.25]
		\draw[thin,lightgray!50] (-1,-1) grid (3,3) ;
		\PieceTangram{TangPetTri} \filldraw[red] (0,0) circle[radius=2pt] ;
		\draw[thick,<->,>=latex] (0,-0.25)--(1,-0.25) node[blue,scale=1.5,midway,below,font=\large\sffamily] {1} ;
		\draw[thick,<->,>=latex] (-0.15,0.15)--++(45:{sqrt(2)}) node[blue,scale=1.5,midway,sloped,above,font=\large\sffamily] {$\mathsf{\sqrt{2}}$} ;
		\draw[thick,<->,>=latex] (1.25,0)--++(0,1) node[blue,scale=1.5,midway,right,font=\large\sffamily] {1} ;
		\draw (1,3) node[purple,scale=1.5,below=1pt,font=\Large\ttfamily] {TangPetTri} ;
	\end{tikzpicture}
	
	\smallskip
	
	\begin{tikzpicture}[scale=1.25]
		\draw[thin,lightgray!50] (-1,-1) grid (3,3) ;
		\PieceTangram{TangCar} \filldraw[red] (0,0) circle[radius=2pt] ;
		\draw[thick,<->,>=latex] (0,-0.25)--(1,-0.25) node[blue,scale=1.5,midway,below,font=\large\sffamily] {1} ;
		\draw[thick,<->,>=latex] (1.25,0)--++(0,1) node[blue,scale=1.5,midway,right,font=\large\sffamily] {1} ;
		\draw[thick,dashed] (0,0)--(-0.65,0.65) (1,1)--(0.35,1.65) ;
		\draw[thick,<->,>=latex] (-0.65,0.65)--++(45:{sqrt(2)}) node[blue,scale=1.5,midway,sloped,above,font=\large\sffamily] {$\mathsf{\sqrt{2}}$} ;
		\draw (1,3) node[purple,scale=1.5,below=1pt,font=\Large\ttfamily] {TangCar} ;
	\end{tikzpicture}
	~
	\begin{tikzpicture}[scale=1.25]
		\draw[thin,lightgray!50] (-1,-1) grid (3,3) ;
		\PieceTangram{TangPara} \filldraw[red] (0,0) circle[radius=2pt] ;
		\draw[thick,<->,>=latex] (0,-0.25)--(1,-0.25) node[blue,scale=1.5,midway,below,font=\large\sffamily] {1} ;
		\draw[thick,<->,>=latex] (-0.15,0.15)--++(45:{sqrt(2)}) node[blue,scale=1.5,midway,sloped,above,font=\large\sffamily] {$\mathsf{\sqrt{2}}$} ;
		\draw[thick,<->,>=latex] (2.15,0)--++(0,1) node[blue,scale=1.5,midway,right,font=\large\sffamily] {1} ;
		\draw[thick,<->,>=latex] (2.15,1.15)--++(135:{0.5*sqrt(2)}) node[blue,midway,sloped,above,font=\large\sffamily] {$\mathsf{0{,}5\sqrt{2}}$} ;
		\draw[thick,dashed] (2,1)--++(0.15,0.15) (1,1)--++(45:{0.5*sqrt(2)+0.2}) ;
		\draw (1,3) node[purple,scale=1.5,below=1pt,font=\Large\ttfamily] {TangPara} ;
	\end{tikzpicture}
\end{center}

Chacune des \textit{pièces} peut donc être :

\begin{itemize}
	\item pivotée, grâce à l'option \tikzlogo{} \texttt{rotate=...} ;
	\item retournée horizontalement ou verticalement, grâce aux options \tikzlogo{} \texttt{xscale=-1} et \texttt{yscale=-1} ;
	\item déplacée, en la plaçant au point de coordonnées \texttt{(x,y)}.
\end{itemize}

Chaque pièce peut posséder un style prédéfini :

\begin{itemize}
	\item \texttt{TangPuzz} : pièce de puzzle, \textit{pleine}, pour laquelle on peut choisir une couleur (\Cle{black} par défaut) ;
	\item \texttt{TangSol} : pièce de puzzle, \textit{avec bordure blanche}, pour laquelle on peut choisir une couleur (\Cle{black} par défaut).
\end{itemize}

\pagebreak

\subsection{Placement des pièces}

Une première manière de placer les \textit{pièces} est donc d'utiliser une syntaxe des \texttt{pic} en \tikzlogo{} :

\begin{PresentationCode}{listing only}
%environnement ou commande tikz
\pic[style,rotate=...,xscale=...,yscale=...] at (x,y) {nom_piece} ;
\end{PresentationCode}

Le package \textsf{TangramTikz} propose également une commande spécifique pour placer les pièces :

\begin{PresentationCode}{listing only}
%environnement ou commande tikz
\PieceTangram[style={couleur}]<xscale=...,yscale=...,rotate=...>(x,y){nom_piece}
\end{PresentationCode}

Un Tangram peut donc être \textit{constuit} manuellement, grâce aux 7 pièces du puzzle, en :

\begin{itemize}
	\item \textit{plaçant} les pièces à l'origine ;
	\item en les \textit{pivotant}/\textit{retournant} pour l'orienter correctement ;
	\item en les \textit{translatant} pour les placer correctement.
\end{itemize}

\begin{PresentationCode}{}
%version corrigée et coloriée (taille par défaut)
\begin{EnvTangramTikz}
	\PieceTangram[TangSol={green}]({0},{0}){TangCar}
	\PieceTangram[TangSol={red}]({-1.5},{1}){TangGrandTri}
	\PieceTangram[TangSol={red}]<rotate=-90>({0.5},{3}){TangGrandTri}
	\PieceTangram[TangSol={purple}]<xscale=-1,rotate=0>({2.5},{2}){TangPara}
	\PieceTangram[TangSol={blue}]({-1.5},{2}){TangPetTri}
	\PieceTangram[TangSol={blue}]<xscale=-1,rotate=90>({-0.5},{2}){TangPetTri}
	\PieceTangram[TangSol={orange}]({-0.5},{3}){TangMoyTri}
	\filldraw[black] (0,0) circle[radius=2pt] ; %repère pour les pièces
\end{EnvTangramTikz}
%version "énoncé"  (taille par défaut)
\begin{EnvTangramTikz}
	\PieceTangram[TangPuzz]({0},{0}){TangCar}
	\PieceTangram[TangPuzz]({-1.5},{1}){TangGrandTri}
	\PieceTangram[TangPuzz]<rotate=-90>({0.5},{3}){TangGrandTri}
	\PieceTangram[TangPuzz]<xscale=-1,rotate=0>({2.5},{2}){TangPara}
	\PieceTangram[TangPuzz]({-1.5},{2}){TangPetTri}
	\PieceTangram[TangPuzz]<xscale=-1,rotate=90>({-0.5},{2}){TangPetTri}
	\PieceTangram[TangPuzz]({-0.5},{3}){TangMoyTri}
\end{EnvTangramTikz}
\end{PresentationCode}

\pagebreak

\section{Fonctionnement \og automatique \fg}

\subsection{Commande}

Un certain nombre de Tangrams sont prédéfinis dans le package \textsf{TangramTikz}, qui peuvent être \textit{appelés} grâce à une commande autonome.

\begin{PresentationCode}{listing only}
%commande autonome pour afficher un Tangram
\TangramTikz[clés]<options tikz>{nom_du_tangram}
\end{PresentationCode}

\begin{PresentationCode}{}
%commande autonome pour afficher le Tangram du Chat/Bateau/Kangourou, avec options par défaut
\TangramTikz{Chat}~~\TangramTikz{Bateau}~~\TangramTikz{Kangourou}
\end{PresentationCode}

\subsection{Clés, options et arguments}

Le premier argument, \textit{optionnel} et entre \texttt{[...]}, correspond aux clés et options :

\begin{itemize}
	\item le booléen \Cle{Puzzle} qui affiche les pièces (monochromes) de puzzle, sans bordure ; \hfill~défaut : \Cle{true}
	\item le booléen \Cle{Correction} qui affiche les pièces (monochromes) du puzzle, avec bordure ; \hfill~défaut : \Cle{false}
	\item \Cle{Couleur} qui paramètre la couleur globale du puzzle avec les booléens précédents ; \hfill~défaut : \Cle{black}
	\item le booléen \Cle{CorrectionCouleur} qui affiche les pièces (colorées) du puzzle, avec bordure ; \hfill~défaut : \Cle{false}
	\item \Cle{ListeCouleurs} qui est la couleur des pièces (\texttt{GT,MT,PT,CAR,PARA})  ;
	
	\hfill~défaut : \Cle{red,orange,blue,green,purple}
	\item \Cle{Sep} qui est l'épaisseur de la bordure des pièces en mode \Cle{Correction} \hfill~défaut : \Cle{1pt}
\end{itemize}

Le deuxième argument, \textit{optionnel} et entre \texttt{<...>}, correspond aux options qui sont passés à l'environnement \tikzlogo{} qui sert de base à la commande, comme par exemple :

\begin{itemize}
	\item un changement d'unité(s), un changement d'échelle ;
	\item une rotation, un alignement vertical ;
	\item etc
\end{itemize}

Le troisième argument, \textit{obligatoire} et entre \texttt{\{...\}} est quant à lui le nom du Tangram issu de la \textit{base de données} présente dans le package (liste ci-après).

\pagebreak

\subsection{Liste des Tangrams inclus dans le package}
%
\texttt{\begin{multicols}{5}
	\begin{itemize}
		\item Carre
		\item Pingouin
		\item Bateau
		\item Maison
		\item Sapin
		\item Chat
		\item Cygne
		\item Pyramide
		\item Canard
		\item Fusee
		\item Bougie
		\item Chemise
		\item Poisson
		\item Voilier
		\item Kangourou
		\item Chien
		\item Lapin
		\item Avion
		\item Coq
		\item Coureur
		\item Danseur
		\item Chameau
		\item Flamant
		\item Coeur
		\item Girafe
		\item Cheval
		\item Chevre
		\item Lion
		\item Usine
		\item Ange
		\item Tour
		\item Ovni
		\item Poule
		\item Tortue
		\item Crabe
		\item Escargot
	\end{itemize}
\end{multicols}}

\medskip

\begin{PresentationCode}{}
\TangramTikz{Fusee}~~
\TangramTikz[Couleur=red]{Fusee}~~
\TangramTikz[Correction]{Fusee}~~
\TangramTikz[Correction,Couleur=lightgray]{Fusee}~~
\TangramTikz[CorrectionCouleur,ListeCouleurs={orange,blue,yellow,green,pink},Sep=1mm]{Fusee}

\TangramTikz<scale=1.5,rotate=30>{Fusee}~~
\TangramTikz<scale=0.75,rotate=-90>{Fusee}
\end{PresentationCode}

\pagebreak

\part{Galerie de Tangrams disponibles}

\begin{PresentationCode}{}
\TangramTikz{Carre}
\TangramTikz[Correction]{Carre}
\TangramTikz[CorrectionCouleur]{Carre}
\end{PresentationCode}

\begin{PresentationCode}{}
\TangramTikz{Pingouin}
\TangramTikz[Correction]{Pingouin}
\TangramTikz[CorrectionCouleur]{Pingouin}
\end{PresentationCode}

\begin{PresentationCode}{}
\TangramTikz{Bateau}
\TangramTikz[Correction]{Bateau}
\TangramTikz[CorrectionCouleur]{Bateau}
\end{PresentationCode}

\begin{PresentationCode}{}
\TangramTikz{Maison}
\TangramTikz[Correction]{Maison}
\TangramTikz[CorrectionCouleur]{Maison}
\end{PresentationCode}

\begin{PresentationCode}{}
\TangramTikz{Sapin}
\TangramTikz[Correction]{Sapin}
\TangramTikz[CorrectionCouleur]{Sapin}
\end{PresentationCode}

\begin{PresentationCode}{}
\TangramTikz{Chat}
\TangramTikz[Correction]{Chat}
\TangramTikz[CorrectionCouleur]{Chat}
\end{PresentationCode}

\begin{PresentationCode}{}
\TangramTikz{Cygne}
\TangramTikz[Correction]{Cygne}
\TangramTikz[CorrectionCouleur]{Cygne}
\end{PresentationCode}

\begin{PresentationCode}{}
\TangramTikz{Pyramide}
\TangramTikz[Correction]{Pyramide}
\TangramTikz[CorrectionCouleur]{Pyramide}
\end{PresentationCode}

\begin{PresentationCode}{}
\TangramTikz{Canard}
\TangramTikz[Correction]{Canard}
\TangramTikz[CorrectionCouleur]{Canard}
\end{PresentationCode}

\begin{PresentationCode}{}
\TangramTikz{Fusee}
\TangramTikz[Correction]{Fusee}
\TangramTikz[CorrectionCouleur]{Fusee}
\end{PresentationCode}

\begin{PresentationCode}{}
\TangramTikz{Bougie}
\TangramTikz[Correction]{Bougie}
\TangramTikz[CorrectionCouleur]{Bougie}
\end{PresentationCode}

\begin{PresentationCode}{}
\TangramTikz{Chemise}
\TangramTikz[Correction]{Chemise}
\TangramTikz[CorrectionCouleur]{Chemise}
\end{PresentationCode}

\begin{PresentationCode}{}
\TangramTikz{Poisson}
\TangramTikz[Correction]{Poisson}
\TangramTikz[CorrectionCouleur]{Poisson}
\end{PresentationCode}

\begin{PresentationCode}{}
\TangramTikz{Voilier}
\TangramTikz[Correction]{Voilier}
\TangramTikz[CorrectionCouleur]{Voilier}
\end{PresentationCode}

\begin{PresentationCode}{}
\TangramTikz{Kangourou}
\TangramTikz[Correction]{Kangourou}
\TangramTikz[CorrectionCouleur]{Kangourou}
\end{PresentationCode}

\begin{PresentationCode}{}
\TangramTikz{Chien}
\TangramTikz[Correction]{Chien}
\TangramTikz[CorrectionCouleur]{Chien}
\end{PresentationCode}

\begin{PresentationCode}{}
\TangramTikz{Lapin}
\TangramTikz[Correction]{Lapin}
\TangramTikz[CorrectionCouleur]{Lapin}
\end{PresentationCode}

\begin{PresentationCode}{}
\TangramTikz{Avion}
\TangramTikz[Correction]{Avion}
\TangramTikz[CorrectionCouleur]{Avion}
\end{PresentationCode}

\begin{PresentationCode}{}
\TangramTikz{Coq}
\TangramTikz[Correction]{Coq}
\TangramTikz[CorrectionCouleur]{Coq}
\end{PresentationCode}

\begin{PresentationCode}{}
\TangramTikz{Coureur}
\TangramTikz[Correction]{Coureur}
\TangramTikz[CorrectionCouleur]{Coureur}
\end{PresentationCode}

\begin{PresentationCode}{}
\TangramTikz{Danseur}
\TangramTikz[Correction]{Danseur}
\TangramTikz[CorrectionCouleur]{Danseur}
\end{PresentationCode}

\begin{PresentationCode}{}
\TangramTikz{Chameau}
\TangramTikz[Correction]{Chameau}
\TangramTikz[CorrectionCouleur]{Chameau}
\end{PresentationCode}

\begin{PresentationCode}{}
\TangramTikz{Flamant}
\TangramTikz[Correction]{Flamant}
\TangramTikz[CorrectionCouleur]{Flamant}
\end{PresentationCode}

\begin{PresentationCode}{}
\TangramTikz{Coeur}
\TangramTikz[Correction]{Coeur}
\TangramTikz[CorrectionCouleur]{Coeur}
\end{PresentationCode}

\begin{PresentationCode}{}
\TangramTikz{Girafe}
\TangramTikz[Correction]{Girafe}
\TangramTikz[CorrectionCouleur]{Girafe}
\end{PresentationCode}

\begin{PresentationCode}{}
\TangramTikz{Cheval}
\TangramTikz[Correction]{Cheval}
\TangramTikz[CorrectionCouleur]{Cheval}
\end{PresentationCode}

\begin{PresentationCode}{}
\TangramTikz{Chevre}
\TangramTikz[Correction]{Chevre}
\TangramTikz[CorrectionCouleur]{Chevre}
\end{PresentationCode}

\begin{PresentationCode}{}
\TangramTikz{Lion}
\TangramTikz[Correction]{Lion}
\TangramTikz[CorrectionCouleur]{Lion}
\end{PresentationCode}

\begin{PresentationCode}{}
\TangramTikz{Usine}
\TangramTikz[Correction]{Usine}
\TangramTikz[CorrectionCouleur]{Usine}
\end{PresentationCode}

\begin{PresentationCode}{}
\TangramTikz{Ange}
\TangramTikz[Correction]{Ange}
\TangramTikz[CorrectionCouleur]{Ange}
\end{PresentationCode}

\begin{PresentationCode}{}
\TangramTikz{Tour}
\TangramTikz[Correction]{Tour}
\TangramTikz[CorrectionCouleur]{Tour}
\end{PresentationCode}

\begin{PresentationCode}{}
\TangramTikz{Ovni}
\TangramTikz[Correction]{Ovni}
\TangramTikz[CorrectionCouleur]{Ovni}
\end{PresentationCode}

\begin{PresentationCode}{}
\TangramTikz{Poule}
\TangramTikz[Correction]{Poule}
\TangramTikz[CorrectionCouleur]{Poule}
\end{PresentationCode}

\begin{PresentationCode}{}
\TangramTikz{Tortue}
\TangramTikz[Correction]{Tortue}
\TangramTikz[CorrectionCouleur]{Tortue}
\end{PresentationCode}

\begin{PresentationCode}{}
\TangramTikz{Crabe}
\TangramTikz[Correction]{Crabe}
\TangramTikz[CorrectionCouleur]{Crabe}
\end{PresentationCode}

\begin{PresentationCode}{}
\TangramTikz{Escargot}
\TangramTikz[Correction]{Escargot}
\TangramTikz[CorrectionCouleur]{Escargot}
\end{PresentationCode}

\newpage

\part{Historique}

\verb|v0.1.5|~:~~~~Nouveaux modèles

\verb|v0.1.4|~:~~~~Nouveaux modèles

\verb|v0.1.3|~:~~~~Nouveaux modèles

\verb|v0.1.2|~:~~~~Nouveaux modèles

\verb|v0.1.1|~:~~~~Nouveaux modèles

\verb|v0.1.0|~:~~~~Version initiale

\end{document}